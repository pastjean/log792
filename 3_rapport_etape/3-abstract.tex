\section*{Résumé}
\addcontentsline{toc}{section}{Résumé}

Le club étudiant Chinook de l'ÉTS afin de continuer son succès en compétition améliore continuellement son véhicule éolien. Un système de contrôle pour l'éolienne tel que celui présent dans les éoliennes statiques doit être mis en place afin d'améliorer la performance du véhicule. Le contrôle de l'angle d'attaque et de la vitesse de rotation de l'éolienne est la façon la plus commune d'ajuster la puissance de sortie d'une éolienne. Afin d'arriver à optimiser le système, l’éolienne du véhicule est d’abord caractérisée expérimentalement puis un système de contrôle de l'angle d'attaque et du ratio de transmission par algorithme génétique est développé, analysé et testé théoriquement et expérimentalement. Le système de contrôle utilisé n'est pas un système conventionnel (PI,PID,etc.) auquel une seule entrée et une seule sortie est considérée, mais plutôt un système dans l'espace d'état dans lequel plusieurs entrées et plusieurs sorties sont considérées.

Résumé des résultats à venir ...
